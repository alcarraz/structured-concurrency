\section{La Funcionalidad Estrella: Fail-Fast}

\subsection{Cancelando tempranamente}
\begin{frame}
    \frametitle{¿Qué pasa cuando una validación falla antes de que las otras terminen de ejecutar?}

    \begin{alertblock}{Reactive Basic: Sin fail-fast}
        \begin{itemize}[<+->]
            \item \texttt{allOf()} espera que TODAS las tareas terminen
            \item Sin cancelación → desperdicio de recursos
            \item ~570ms incluso si falla en 200ms
        \end{itemize}
    \end{alertblock}

    \onslide<+->
    \begin{exampleblock}{Reactive "Fixed": Fail-fast manual}
        \begin{itemize}[<+->]
            \item Se puede lograr $\ldots$ pero requiere lógica compleja
            \item \texttt{failFast.completeExceptionally()} + coordinación manual
        \end{itemize}
    \end{exampleblock}

    \onslide<+->
    \begin{block}{Structured: Fail-fast automático}
        \begin{itemize}[<+->]
            \item \texttt{StructuredTaskScope.open()} = fail-fast por defecto
            \item Cancelación automática en cascada
        \end{itemize}
    \end{block}

    \begin{center}
        \Large{\textcolor{red}{$\blacktriangleright$ La demo más importante...}}
    \end{center}
\end{frame}

\subsection{Por Qué Esto Importa}
\begin{frame}
    \begin{exampleblock}{Impacto en la Experiencia de Usuario}
        \begin{itemize}[<+->]
            \item \textbf{Falla rápido}: Reporta errores inmediatamente, mejor UX
            \item \textbf{Menor latencia} en casos de error
            \item \textbf{Feedback claro}: Usuario no espera innecesariamente
        \end{itemize}
    \end{exampleblock}

    \onslide<+->
    \begin{block}{Impacto en Recursos y Costos}
        \begin{itemize}[<+->]
            \item \textbf{Eficiencia}: Cancela trabajo innecesario automáticamente
            \item \textbf{Ahorro de CPU}: No procesa validaciones que no se usarán.
            \item \textbf{Menor carga}: Sistema responde más rápido en picos de errores
        \end{itemize}
    \end{block}

\end{frame}

