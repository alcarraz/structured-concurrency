\section{Conclusiones}

\subsection{¿Por Qué Adoptar Concurrencia Estructurada?}
\begin{frame}
\begin{exampleblock}{Beneficios Inmediatos}
\begin{itemize}
\item \textbf{Legibilidad:} Código que lee como se ejecuta
\item \textbf{Mantenibilidad:} Más fácil de modificar y extender
\item \textbf{Debugging:} Stack traces claros y útiles
\item \textbf{Performance:} Misma velocidad, mejor código
\item \textbf{Resource Safety:} Garantías automáticas de cleanup
\end{itemize}
\end{exampleblock}

\begin{alertblock}{$\star$ Roadmap de Adopción}
\begin{itemize}
\item \textbf{HOY (Producción):} Scoped Values (\textcolor{green}{\textbf{ESTABLES}} en Java 25)
\item \textbf{AHORA (Experimentar):} Concurrencia Estructurada (quinta preview - muy maduro)
\item \textbf{Java 26:} Concurrencia Estructurada estable (alta probabilidad)
\end{itemize}
\end{alertblock}
\end{frame}

\subsection{Mensajes Clave}
\begin{frame}
\begin{enumerate}
\item \large{\textbf{Scoped Values ya son estables - adopta hoy}}
\item \large{\textbf{De vuelta a la simplicidad sin sacrificar performance}}
\item \large{\textbf{Concurrencia Estructurada aún está en preview}}
\item \large{\textbf{Código que lee como se ejecuta}}
\item \large{\textbf{Mejor developer experience sin trade-offs}}
\end{enumerate}

\begin{center}
\textbf{GitHub:} \url{https://github.com/alcarraz/structured-concurrency}
\end{center}
\end{frame}


