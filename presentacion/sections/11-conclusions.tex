\section{Conclusiones}

\subsection{¿Por qué adoptar concurrencia estructurada?}
\begin{frame}
\begin{exampleblock}{Beneficios Inmediatos}
\begin{itemize}
\item \textbf{Legibilidad:} Código que lee como se ejecuta
\item \textbf{Mantenibilidad:} Más fácil de modificar y extender
\item \textbf{Depuración:} Stack traces claros y útiles
\item \textbf{Rendimiento:} Misma velocidad, mejor código
\item \textbf{Manejo de recursos:} Garantías automáticas de cleanup
\end{itemize}
\end{exampleblock}

\begin{block}{$\star$ Roadmap de Adopción}
\begin{itemize}
\item \textbf{HOY (Producción):} Scoped Values (\textcolor{green}{\textbf{ESTABLES}} en Java 25)
\item \textbf{AHORA (Experimentar):} Concurrencia Estructurada (quinta preview)
\item \textbf{Java 26:} Aún en preview (sexta, \link{https://openjdk.org/jeps/525}{JEP 525}).  
\item \textbf{Java 29 (Siguiente LTS):} Concurrencia Estructurada estable \emoji{crossed-fingers}
\end{itemize}
\end{block}
\end{frame}

\subsection{Mensajes Clave}
\begin{frame}
\begin{enumerate}
\item \large{\textbf{Scoped Values ya son estables:} Empieza a adoptarlos}
\item \large{\textbf{De vuelta a la simplicidad sin sacrificar rendimiento}}
\item \large{\textbf{Concurrencia Estructurada aún está en preview}}
\item \large{\textbf{Código que lee como se ejecuta}}
\item \large{\textbf{Mejor experiencia del desarrollador sin trade-offs}}
\end{enumerate}

\begin{center}
\textbf{GitHub:} \url{https://github.com/alcarraz/structured-concurrency}
\end{center}
\end{frame}


